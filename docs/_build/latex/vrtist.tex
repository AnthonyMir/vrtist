%% Generated by Sphinx.
\def\sphinxdocclass{report}
\documentclass[letterpaper,10pt,english,openany,oneside]{sphinxmanual}
\ifdefined\pdfpxdimen
   \let\sphinxpxdimen\pdfpxdimen\else\newdimen\sphinxpxdimen
\fi \sphinxpxdimen=.75bp\relax
\ifdefined\pdfimageresolution
    \pdfimageresolution= \numexpr \dimexpr1in\relax/\sphinxpxdimen\relax
\fi
%% let collapsable pdf bookmarks panel have high depth per default
\PassOptionsToPackage{bookmarksdepth=5}{hyperref}

\PassOptionsToPackage{warn}{textcomp}
\usepackage[utf8]{inputenc}
\ifdefined\DeclareUnicodeCharacter
% support both utf8 and utf8x syntaxes
  \ifdefined\DeclareUnicodeCharacterAsOptional
    \def\sphinxDUC#1{\DeclareUnicodeCharacter{"#1}}
  \else
    \let\sphinxDUC\DeclareUnicodeCharacter
  \fi
  \sphinxDUC{00A0}{\nobreakspace}
  \sphinxDUC{2500}{\sphinxunichar{2500}}
  \sphinxDUC{2502}{\sphinxunichar{2502}}
  \sphinxDUC{2514}{\sphinxunichar{2514}}
  \sphinxDUC{251C}{\sphinxunichar{251C}}
  \sphinxDUC{2572}{\textbackslash}
\fi
\usepackage{cmap}
\usepackage[T1]{fontenc}
\usepackage{amsmath,amssymb,amstext}
\usepackage{babel}



\usepackage{tgtermes}
\usepackage{tgheros}
\renewcommand{\ttdefault}{txtt}



\usepackage[Bjarne]{fncychap}
\usepackage{sphinx}

\fvset{fontsize=auto}
\usepackage{geometry}


% Include hyperref last.
\usepackage{hyperref}
% Fix anchor placement for figures with captions.
\usepackage{hypcap}% it must be loaded after hyperref.
% Set up styles of URL: it should be placed after hyperref.
\urlstyle{same}


\usepackage{sphinxmessages}
\setcounter{tocdepth}{1}



\title{VRtist}
\date{Sep 16, 2021}
\release{v0.1.1}
\author{Ubisoft}
\newcommand{\sphinxlogo}{\vbox{}}
\renewcommand{\releasename}{Release}
\makeindex
\begin{document}

\pagestyle{empty}
\sphinxmaketitle
\pagestyle{plain}
\sphinxtableofcontents
\pagestyle{normal}
\phantomsection\label{\detokenize{index::doc}}



\chapter{Introduction}
\label{\detokenize{index:introduction}}
\sphinxAtStartPar
Directors and artists can setup a 3D scene being immersed into the virtual world.
That will allow them to have a better understanding of the 3D scene.


\section{Disclaimer}
\label{\detokenize{index:disclaimer}}
\sphinxAtStartPar
This project is in alpha state and actively developed. Do not use it to edit your production assets without a backup or you might break them.


\section{Main Features}
\label{\detokenize{index:main-features}}\begin{itemize}
\item {} 
\sphinxAtStartPar
Set dressing: import 3D objects and compose the virtual set.

\item {} 
\sphinxAtStartPar
Camera: naturally move around and find the best camera angles and depth of field.

\item {} 
\sphinxAtStartPar
Animation: use the record mode or key frames to create animations.

\item {} 
\sphinxAtStartPar
Lighting: use gradient sky or fine\sphinxhyphen{}tune the lighting using lights (directional, point and cone).

\item {} 
\sphinxAtStartPar
Nonlinear editing: nonlinear editing using multi\sphinxhyphen{}camera footages.

\item {} 
\sphinxAtStartPar
Live link: Blender and VRtist scene live link.

\end{itemize}


\section{Supported VR Devices}
\label{\detokenize{index:supported-vr-devices}}
\sphinxAtStartPar
For now, VRtist has only been tested with the Oculus Rift S and Oculus Quest devices.


\section{License and copyright}
\label{\detokenize{index:license-and-copyright}}
\sphinxAtStartPar
The original code is Copyright (C) 202 Ubisoft.

\sphinxAtStartPar
All code of the VRtist project is under the MIT license.

\sphinxAtStartPar
All 3D models (FBX files or Unity prefabs) are licensed under the Creative Commons Attribution\sphinxhyphen{}NonCommercial\sphinxhyphen{}NoDerivatives 4.0 International License.
To view a copy of this license, visit \sphinxurl{http://creativecommons.org/licenses/by-nc-nd/4.0/} or send a letter to Creative Commons, PO Box2866, Mountain View, CA 94042, USA.


\chapter{Getting Started}
\label{\detokenize{index:getting-started}}

\section{Getting Started}
\label{\detokenize{Start/GettingStarted:getting-started}}\label{\detokenize{Start/GettingStarted::doc}}

\subsection{Download project}
\label{\detokenize{Start/GettingStarted:download-project}}
\sphinxAtStartPar
Open the \sphinxhref{https://github.com/ubisoft/vrtist}{latest release page}  from the Mixer Gihub \sphinxhref{https://github.com/ubisoft/vrtist/releases}{releases page}.


\subsection{Install \& Launch}
\label{\detokenize{Start/GettingStarted:install-launch}}
\sphinxAtStartPar
Just unzip the release file. Then launch the VRtist.exe and put your headset.


\subsection{How to build}
\label{\detokenize{Start/GettingStarted:how-to-build}}
\sphinxAtStartPar
VRtist is using Unity version: 2020.2.1f1

\sphinxAtStartPar
For now, VRtist has been tested on Windows 64bit only. Build the “Main” scene in the Unity editor:
\begin{itemize}
\item {} 
\sphinxAtStartPar
Platform: PC, Mac \& Linux Standalone

\item {} 
\sphinxAtStartPar
Target Platform: Windows

\item {} 
\sphinxAtStartPar
Architecture: x86\_64

\end{itemize}


\subsubsection{Dependencies}
\label{\detokenize{Start/GettingStarted:dependencies}}
\sphinxAtStartPar
VRtist uses the following libraries as DLLs:
\begin{itemize}
\item {} 
\sphinxAtStartPar
Assimp: \sphinxurl{http://www.assimp.org}

\item {} 
\sphinxAtStartPar
OpenImageIO: \sphinxurl{http://www.openimageio.org}

\end{itemize}


\subsubsection{Settings \& Logs}
\label{\detokenize{Start/GettingStarted:settings-logs}}
\sphinxAtStartPar
On Windows OS, VRtist writes settings, logs and saves to the \%userprofile\%/AppData/LocalLow/Ubisoft/VRtist/ directory (Unity.Application.persistentDataPath).


\subsection{Asset Bank}
\label{\detokenize{Start/GettingStarted:asset-bank}}
\sphinxAtStartPar
VRtist is ditributed with a predefined set of 3D objects. It also supports FBX files import from a specified directory (default: D:VRtistData). This can be overriden in the advanced settings. FBX files may be exported from Blender using the following options:
\begin{itemize}
\item {} 
\sphinxAtStartPar
scale: 0.01

\item {} 
\sphinxAtStartPar
Y Forward

\item {} 
\sphinxAtStartPar
Z Up

\item {} 
\sphinxAtStartPar
Apply Unit: unchecked

\item {} 
\sphinxAtStartPar
Apply Transform: checked is advised

\end{itemize}


\section{Live Link with Blender}
\label{\detokenize{Start/Link:live-link-with-blender}}\label{\detokenize{Start/Link::doc}}
\sphinxAtStartPar
It’s possible to sync a Blender scene with a VRtist one via the Mixer add\sphinxhyphen{}on. A typical use case is to have a working Blender scene and to explore and modify it in VR with VRtist.


\subsection{Install Blender}
\label{\detokenize{Start/Link:install-blender}}
\sphinxAtStartPar
Download Blender 2.91 or above from blender.org (either install version or portable version).


\subsection{Install Mixer}
\label{\detokenize{Start/Link:install-mixer}}
\sphinxAtStartPar
VRtist uses Mixer to get the content of a Blender scene and to synchronize it with Unity.

\sphinxAtStartPar
see: \sphinxurl{https://github.com/ubisoft/mixer}

\sphinxAtStartPar
Mixer is a standard Blender addon. Install it like any other addon.
\begin{itemize}
\item {} 
\sphinxAtStartPar
Go to Edit \textgreater{} Preferences \textgreater{} Add\sphinxhyphen{}ons

\item {} 
\sphinxAtStartPar
Click the install… button and search “mixer.zip”

\item {} 
\sphinxAtStartPar
Activate the add\sphinxhyphen{}on by checking the “Collaboration Mixer” item

\end{itemize}


\subsection{How to Launch the Live Link}
\label{\detokenize{Start/Link:how-to-launch-the-live-link}}
\sphinxAtStartPar
Open your scene in Blender. In the 3D Viewport, press “N” to open the Mixer addon. Enter the VRtist.exe path.

\sphinxAtStartPar
Launch or join a server, create or join a room and press “Launch VRtist”. Put your headset and enjoy!

\sphinxAtStartPar
A Windows firewall popup may appear during the first launch, accept it.

\sphinxAtStartPar
When done, everything that you did in VRtist is in your Blender scene. You can continue to work on it and relaunch VRtist anytime you want.


\chapter{Basics}
\label{\detokenize{index:basics}}

\section{Basics}
\label{\detokenize{Basics/Basics:basics}}\label{\detokenize{Basics/Basics::doc}}

\subsection{Controllers}
\label{\detokenize{Basics/Basics:controllers}}
\sphinxAtStartPar
The primary controller is exclusively used for the palette, the secondary controller is used for tools and scene interaction. Current active tool is shown by the mouthpiece on the secondary controller.

\sphinxAtStartPar
By default, the primary controller is left, and secondary is right.

\sphinxAtStartPar
The controllers have some basic inputs.
The two buttons on the primary controller are used to undo or redo the last command.
The two buttons on the secondary controller are used to duplicate the selected item, and quickly switch tools. The tools switch between selection (blue mouthpiece), deletion (red mouthpiece) and the last used tool (tool’s mouthpiece).

\sphinxAtStartPar
Primary and secondary controllers can be switched in the user settings.


\subsection{Lobby}
\label{\detokenize{Basics/Basics:lobby}}
\noindent{\hspace*{\fill}\sphinxincludegraphics[width=300\sphinxpxdimen]{{Lobby}.png}\hspace*{\fill}}

\sphinxAtStartPar
When launching Vrtist, the user will arrive in the Lobby.
Previously saved scenes will be displayed on the screen facing the user.
The user can select one scene to load and edit.
Click “new project” to create a new project. This will load a blank scene.


\subsection{Panels}
\label{\detokenize{Basics/Basics:panels}}
\sphinxAtStartPar
Some features of VrTist use floating panels, all panels can be moved by dragging the top part of the panel. Floating panels position is relative to the user’s space.
Panels can be snapped together by dragging one next to another, arrows will appear around the other panel.


\section{Movements}
\label{\detokenize{Basics/Movements:movements}}\label{\detokenize{Basics/Movements::doc}}
\noindent{\hspace*{\fill}\sphinxincludegraphics[width=300\sphinxpxdimen]{{paletteMovement}.png}\hspace*{\fill}}

\begin{DUlineblock}{0em}
\item[] 
\end{DUlineblock}

\sphinxAtStartPar
Multiple movement systems can be used in Vrtist.


\subsection{Default}
\label{\detokenize{Basics/Movements:default}}
\begin{DUlineblock}{0em}
\item[] The default movement system allows the user to move the world around him.
\item[] By moving the two controllers and moving them in the same direction, the world will be dragged in that direction.
\item[] By moving the two controllers around each other, the world will rotate around the user.
\item[] By moving the two controllers closer or apart from each other, the scale of the world will be reduced or increased.
\item[] The current scale of the world is displayed on the primary controller.
\end{DUlineblock}


\subsection{Teleport}
\label{\detokenize{Basics/Movements:teleport}}
\noindent{\hspace*{\fill}\sphinxincludegraphics[width=300\sphinxpxdimen]{{paletteMovement1}.png}\hspace*{\fill}}

\begin{DUlineblock}{0em}
\item[] 
\end{DUlineblock}

\begin{DUlineblock}{0em}
\item[] The teleport system allows the user to teleport at a targeted spot using the primary controller.
\item[] Move the joystick up to show the teleport target and aim at the wanted position. Then rotate the joystick to move the arrow on the target and select the wanted direction.
\item[] Release the joystick to teleport.
\item[] You can also move the joystick left and right to rotate without moving.
\end{DUlineblock}


\subsection{Orbit}
\label{\detokenize{Basics/Movements:orbit}}
\noindent{\hspace*{\fill}\sphinxincludegraphics[width=300\sphinxpxdimen]{{paletteMovement2}.png}\hspace*{\fill}}

\begin{DUlineblock}{0em}
\item[] 
\end{DUlineblock}

\begin{DUlineblock}{0em}
\item[] The orbit system allows the user to spin around an object.
\item[] Aim at an object in the scene, once the desired object is highlighted, press the grip, move the joystick to move around the object.
\item[] Use the options in the palette to change scale speed, move speed or orbit speed.
\end{DUlineblock}


\subsection{FPS}
\label{\detokenize{Basics/Movements:fps}}
\noindent{\hspace*{\fill}\sphinxincludegraphics[width=300\sphinxpxdimen]{{paletteMovement3}.png}\hspace*{\fill}}

\begin{DUlineblock}{0em}
\item[] 
\end{DUlineblock}

\begin{DUlineblock}{0em}
\item[] The fps system allows the user to move on an horizontal plane using both controllers’ joysticks.
\item[] Primary controller joystick to move forward, backwards and straf left and right. Secondary controller to rotate.
\end{DUlineblock}


\subsection{Drone}
\label{\detokenize{Basics/Movements:drone}}
\noindent{\hspace*{\fill}\sphinxincludegraphics[width=300\sphinxpxdimen]{{paletteMovement4}.png}\hspace*{\fill}}

\begin{DUlineblock}{0em}
\item[] 
\end{DUlineblock}

\begin{DUlineblock}{0em}
\item[] The drone system allows the user to move like a drone, using both joysticks.
\item[] The primary controller joystick up and down axis moves the user up or down.
\item[] The primary controller joystick left and right axis rotates the user.
\item[] The secondary controller moves the user horizontally.
\end{DUlineblock}


\subsection{Free fly}
\label{\detokenize{Basics/Movements:free-fly}}
\noindent{\hspace*{\fill}\sphinxincludegraphics[width=300\sphinxpxdimen]{{paletteMovement5}.png}\hspace*{\fill}}

\begin{DUlineblock}{0em}
\item[] 
\end{DUlineblock}

\begin{DUlineblock}{0em}
\item[] Free fly navigation allows the user to move using the primary controller.
\item[] The joystick up and down axis moves the user forward and backward.
\item[] The joystick left and right  axis and grip rotates the user.
\end{DUlineblock}


\section{Settings}
\label{\detokenize{Basics/Settings:settings}}\label{\detokenize{Basics/Settings::doc}}

\subsection{Save and load}
\label{\detokenize{Basics/Settings:save-and-load}}
\noindent{\hspace*{\fill}\sphinxincludegraphics[width=300\sphinxpxdimen]{{PaletteSettings}.png}\hspace*{\fill}}

\begin{DUlineblock}{0em}
\item[] 
\item[] The settings panel contains multiple sections.
\item[] The Save and load section allows the user to save the current state of the scene or reload the scene to the previous saved state.
\item[] It also contains options to export to Universal Scene Description (USD) and USD animation. (Experimental feature)
\end{DUlineblock}


\subsection{Display options}
\label{\detokenize{Basics/Settings:display-options}}
\noindent{\hspace*{\fill}\sphinxincludegraphics[width=300\sphinxpxdimen]{{PaletteSettings1}.png}\hspace*{\fill}}

\begin{DUlineblock}{0em}
\item[] 
\item[] The display section allows the user to choose what he wants to display in the user interface.
\item[] The gizmos and locators options are also available in the top part of the palette.
\end{DUlineblock}


\subsection{Sound options}
\label{\detokenize{Basics/Settings:sound-options}}
\noindent{\hspace*{\fill}\sphinxincludegraphics[width=300\sphinxpxdimen]{{PaletteSettings2}.png}\hspace*{\fill}}

\begin{DUlineblock}{0em}
\item[] 
\item[] The sound section allows the user to set the level of various sounds.
\end{DUlineblock}


\subsection{Advanced options}
\label{\detokenize{Basics/Settings:advanced-options}}
\noindent{\hspace*{\fill}\sphinxincludegraphics[width=300\sphinxpxdimen]{{PaletteSettings3}.png}\hspace*{\fill}}

\begin{DUlineblock}{0em}
\item[] 
\item[] The advanced options allow the user to define the Asset bank directory, switch primary and secondary controllers, keep the palette open when the trigger is released, display fps on the secondary controller, and show the console window.
\end{DUlineblock}


\subsection{Video Output options}
\label{\detokenize{Basics/Settings:video-output-options}}
\noindent{\hspace*{\fill}\sphinxincludegraphics[width=300\sphinxpxdimen]{{PaletteSettings4}.png}\hspace*{\fill}}

\begin{DUlineblock}{0em}
\item[] 
\item[] The video Output Options allow the user to define the quality of video outputs, and destination folder.
\end{DUlineblock}


\chapter{Tools}
\label{\detokenize{index:tools}}

\section{Palette}
\label{\detokenize{Tools/Palette:palette}}\label{\detokenize{Tools/Palette::doc}}
\begin{DUlineblock}{0em}
\item[] The palette contains every tool available to the user. To open it, use the trigger on the primary controller.
\item[] Some options are available at the top of the palette.
\end{DUlineblock}

\begin{DUlineblock}{0em}
\item[] Click the \sphinxincludegraphics[width=23\sphinxpxdimen]{{home}.png} icon to return to the lobby.
\item[] 
\item[] Click the \sphinxincludegraphics[width=23\sphinxpxdimen]{{save}.png} icon opens the save and load setting panel.
\item[] 
\item[] Click the \sphinxincludegraphics[width=23\sphinxpxdimen]{{player_play}.png} icon, to play scene animations.
\item[] 
\item[] Click the \sphinxincludegraphics[width=23\sphinxpxdimen]{{show}.png} icon to show/hide gizmos in the scene.
\item[] 
\item[] Click the \sphinxincludegraphics[width=23\sphinxpxdimen]{{axis}.png} icon to show/hide locators in the scene.
\end{DUlineblock}

\begin{DUlineblock}{0em}
\item[] The palette can be pinned around the user. To do so, click on the \sphinxincludegraphics[width=23\sphinxpxdimen]{{pin}.png} icon, and move the palette to a new position.
\item[] The palette position is saved between executions.
\end{DUlineblock}


\section{Selection}
\label{\detokenize{Tools/Selection:selection}}\label{\detokenize{Tools/Selection::doc}}
\noindent{\hspace*{\fill}\sphinxincludegraphics[width=300\sphinxpxdimen]{{paletteSelection}.png}\hspace*{\fill}}

\begin{DUlineblock}{0em}
\item[] 
\item[] The selection panel is used to select or unselect objects.
\item[] To select an object, move the secondary controller inside an object and press the trigger.
\item[] To unselect an item, press the trigger in an empty space.
\end{DUlineblock}


\subsection{Select panel}
\label{\detokenize{Tools/Selection:select-panel}}
\begin{DUlineblock}{0em}
\item[] The first sub panel displays information about the selection, and different options.
\item[] To use the selection tool, click on the lower \sphinxincludegraphics[width=23\sphinxpxdimen]{{select}.png} icon. This is the default tool. When the selection tool is active the mouthpiece will be blue.
\item[] To use the deletion tool click the \sphinxincludegraphics[width=23\sphinxpxdimen]{{eraser}.png} icon and select an object. When the deletion tool is active the mouthpiece will be red.
\item[] Delete all selected items with the “Delete Selection” button.
\item[] To deform an item, after ticking the “Deform the selected items” checkbox, move the secondary controller to one side of the object’s selection box, and press the trigger to drag that side. To deform the object uniformly along all sides, tick the “Uniform scale” checkbox.
\item[] Ticking Snap will make the selection box of moved objects stick to the selection box of a close object. The snapping detection can be limited to surfaces facing upwards with the “snap to ground only” checkbox.
\end{DUlineblock}


\subsection{Transform panel}
\label{\detokenize{Tools/Selection:transform-panel}}
\noindent{\hspace*{\fill}\sphinxincludegraphics[width=300\sphinxpxdimen]{{paletteSelection1}.png}\hspace*{\fill}}

\begin{DUlineblock}{0em}
\item[] 
\item[] The second sub panel displays the transform of the object, and allows edits.
\item[] Values can be edited by clicking on the number label.
\item[] To reset a position, rotation or scale to its default value, use the \sphinxincludegraphics[width=23\sphinxpxdimen]{{reset}.png} icon on the corresponding parameter. For position and rotation, the default value is 0, for scale the value is 1.
\item[] To lock a parameter, click on the \sphinxincludegraphics[width=23\sphinxpxdimen]{{locked}.png} icon.
\end{DUlineblock}

\begin{DUlineblock}{0em}
\item[] To add a parent or a “look at” constraint to an object, click on the \sphinxincludegraphics[width=23\sphinxpxdimen]{{dof}.png} icon in front of the desired constraint, and select the constraining object.
\item[] The name of the constraining object will appear next to the constraint.
\item[] To remove a constraint, click on the \sphinxincludegraphics[width=23\sphinxpxdimen]{{trash}.png} icon next to it.
\end{DUlineblock}


\section{Painting}
\label{\detokenize{Tools/Painting:painting}}\label{\detokenize{Tools/Painting::doc}}
\noindent{\hspace*{\fill}\sphinxincludegraphics[width=300\sphinxpxdimen]{{PalettePainting}.png}\hspace*{\fill}}

\begin{DUlineblock}{0em}
\item[] 
\item[] The painting tool is used to paint or draw on the environment.
\item[] Use the color selector panel to change the color.
\end{DUlineblock}

\noindent{\hspace*{\fill}\sphinxincludegraphics[width=300\sphinxpxdimen]{{ColorPicker}.png}\hspace*{\fill}}

\begin{DUlineblock}{0em}
\item[] 
\item[] There are multiple types of painting tools:
\item[] 
\item[] The drawing tube tool draws tubes with the secondary controller. Check “Project paint on surface” for the tube to be projected on the edges of an object.
\item[] 
\end{DUlineblock}

\noindent{\hspace*{\fill}\sphinxincludegraphics[width=300\sphinxpxdimen]{{PalettePainting0}.png}\hspace*{\fill}}

\begin{DUlineblock}{0em}
\item[] 
\item[] The paint ribbon tool draws ribbons instead of tubes.
\item[] 
\end{DUlineblock}

\noindent{\hspace*{\fill}\sphinxincludegraphics[width=300\sphinxpxdimen]{{PalettePainting1}.png}\hspace*{\fill}}

\begin{DUlineblock}{0em}
\item[] 
\item[] The paint hull tool creates a volume whose edges follow the controller.
\item[] 
\end{DUlineblock}

\noindent{\hspace*{\fill}\sphinxincludegraphics[width=300\sphinxpxdimen]{{PalettePainting2}.png}\hspace*{\fill}}

\begin{DUlineblock}{0em}
\item[] 
\item[] The paint volum tool creates a volume by making cells that will link to other close cells of the same drawing.
\item[] 
\end{DUlineblock}

\noindent{\hspace*{\fill}\sphinxincludegraphics[width=300\sphinxpxdimen]{{PalettePainting3}.png}\hspace*{\fill}}


\section{Lighting}
\label{\detokenize{Tools/Lighting:lighting}}\label{\detokenize{Tools/Lighting::doc}}
\noindent{\hspace*{\fill}\sphinxincludegraphics[width=300\sphinxpxdimen]{{PaletteLighting}.png}\hspace*{\fill}}

\begin{DUlineblock}{0em}
\item[] 
\item[] The lighting tool is used to add lights in the scene. Three different types of lights are available to the user. To add a light, drag the light’s model into the scene.
\end{DUlineblock}

\begin{DUlineblock}{0em}
\item[] Point light adds a light at a point in the scene that projects light in every direction.
\end{DUlineblock}

\begin{DUlineblock}{0em}
\item[] Spot light adds a light at a point in the scene that projects light restricted to an angle.
\end{DUlineblock}

\begin{DUlineblock}{0em}
\item[] Sun light adds a global light in the scene that projects in a single direction.
\end{DUlineblock}

\begin{DUlineblock}{0em}
\item[] Use the panel on the right to select a light already in the scene.
\end{DUlineblock}

\noindent{\hspace*{\fill}\sphinxincludegraphics[width=300\sphinxpxdimen]{{PaletteLighting1}.png}\hspace*{\fill}}

\begin{DUlineblock}{0em}
\item[] 
\item[] When a light is selected, new sliders to modify its parameters will appear.
\item[] 
\end{DUlineblock}

\noindent{\hspace*{\fill}\sphinxincludegraphics[width=300\sphinxpxdimen]{{PaletteLighting2}.png}\hspace*{\fill}}


\section{Camera}
\label{\detokenize{Tools/Camera:camera}}\label{\detokenize{Tools/Camera::doc}}
\noindent{\hspace*{\fill}\sphinxincludegraphics[width=300\sphinxpxdimen]{{paletteCamera0}.png}\hspace*{\fill}}

\begin{DUlineblock}{0em}
\item[] 
\item[] The camera tool is used to add a camera in the scene.
\item[] To add a new camera, drag the camera model into the scene. A panel displaying the camera’s view will appear showing the point of view of the camera at the position of the camera.
\item[] 
\item[] The right panel displays all cameras present in the scene.
\item[] 
\item[] The middle panel displays available options for the selected camera.
\item[] The focal and damping can be directly edited using the sliders.
\item[] The focal can also be changed using the slider at the bottom of the camera view panel.
\item[] 
\item[] To edit the depth of field, first click the checkbox.
\item[] The Focus and aperture can be defined using the sliders.
\item[] To define the focus distance, drag the depth of field sphere to the desired distance.
\item[] 
\end{DUlineblock}

\noindent{\hspace*{\fill}\sphinxincludegraphics[width=300\sphinxpxdimen]{{CameraFocus}.png}\hspace*{\fill}}

\begin{DUlineblock}{0em}
\item[] 
\item[] The focus point can also be defined by clicking the target icon on the camera view panel, and selecting the focus point in the camera view.
\item[] 
\end{DUlineblock}

\noindent{\hspace*{\fill}\sphinxincludegraphics[width=300\sphinxpxdimen]{{CameraView}.png}\hspace*{\fill}}

\begin{DUlineblock}{0em}
\item[] 
\item[] On the camera view panel, click the \sphinxincludegraphics[width=23\sphinxpxdimen]{{snapshot}.png} icon to create a snapshot, click the \sphinxincludegraphics[width=23\sphinxpxdimen]{{record_video}.png} icon to record and export a video.
\item[] 
\item[] The camera can be constrained to other objects. Select the camera panel, and use the selection tool to add a parent or “look at” constraint.
\item[] In the same way, the focus point can be constrained to another object by selecting the target sphere and adding a parent constraint.
\end{DUlineblock}


\section{Material}
\label{\detokenize{Tools/Material:material}}\label{\detokenize{Tools/Material::doc}}
\noindent{\hspace*{\fill}\sphinxincludegraphics[width=300\sphinxpxdimen]{{PaletteMaterial}.png}\hspace*{\fill}}

\begin{DUlineblock}{0em}
\item[] 
\item[] The material tool is used to edit the appearance of an object.
\item[] 
\item[] To set the material of an object, customize the color, roughness and metallic level of the material using the color selector and the sliders. The sphere shows a preview of the material. Click the “set material” button, then select one or multiple objects in the scene to apply the material.
\item[] To update a material on an already selected object, click the “Update Selection material” button, and customize the material. The selected item will be updated with the new material.
\item[] To copy the material of an object, select an object, then click the Pick Material button, this will update the values of the material with the ones from the object. The material can now be set on other objects with “Set Material”.
\end{DUlineblock}


\section{AssetBank}
\label{\detokenize{Tools/AssetBank:assetbank}}\label{\detokenize{Tools/AssetBank::doc}}
\noindent{\hspace*{\fill}\sphinxincludegraphics[width=300\sphinxpxdimen]{{PaletteAssetBank}.png}\hspace*{\fill}}

\begin{DUlineblock}{0em}
\item[] 
\item[] The asset bank tool is used to add objects to the scene.
\item[] To add an object, target it with the laser, then use the grip to drag it into the scene.
\item[] Used the icon at the bottom of the asset panel to browse all available assets.
\item[] Use the secondary controller with the grip to place the object. Change the size of the object by moving the joystick up and down on the secondary controller.
\item[] 
\item[] Use the duplicate button instead of releasing the grip to add multiple instances of the same object.
\item[] 
\item[] The assets available in the asset bank, are a list of predefined objects. To add an object to the asset bank, add the .fbx file into the default asset folder.
\item[] The folder location can be found in the settings screen.
\end{DUlineblock}


\section{Skybox}
\label{\detokenize{Tools/Skybox:skybox}}\label{\detokenize{Tools/Skybox::doc}}
\noindent{\hspace*{\fill}\sphinxincludegraphics[width=300\sphinxpxdimen]{{paletteSkybox}.png}\hspace*{\fill}}

\begin{DUlineblock}{0em}
\item[] 
\item[] The skybox tool is used to change the color of the skybox.
\item[] To edit the color of the top, middle, bottom part of the skybox, select the button, then the color on the color selector.
\item[] Presets can be made for skybox settings. Click “Save as new” to create a new preset. “Save current” to overwrite the current preset. Use the panel on the right to select saved presets.
\item[] 
\end{DUlineblock}

\noindent{\hspace*{\fill}\sphinxincludegraphics[width=300\sphinxpxdimen]{{paletteSkybox1}.png}\hspace*{\fill}}

\begin{DUlineblock}{0em}
\item[] 
\item[] Use the icons at the bottom of the panel to browse available presets.
\item[] To create a copy of a preset click the \sphinxincludegraphics[width=23\sphinxpxdimen]{{duplicate}.png} icon, to delete a saved preset click the \sphinxincludegraphics[width=23\sphinxpxdimen]{{trash}.png} icon.
\end{DUlineblock}


\section{Gun}
\label{\detokenize{Tools/Gun:gun}}\label{\detokenize{Tools/Gun::doc}}
\noindent{\hspace*{\fill}\sphinxincludegraphics[width=300\sphinxpxdimen]{{palettePhysics}.png}\hspace*{\fill}}

\begin{DUlineblock}{0em}
\item[] 
\item[] The gun tool is used to shoot objects with physics into the scene.
\item[] Select one or multiple objects in the asset list to add them to the gun.
\item[] Use the slider to define fire rate, power and scale. Use the secondary controller trigger to fire the objects from the secondary controller. If multiple objects are added to the gun, they will be fired one by one in the selected order.
\end{DUlineblock}


\chapter{Advanced Features}
\label{\detokenize{index:advanced-features}}

\section{Animation Engine}
\label{\detokenize{Advanced/Animation:animation-engine}}\label{\detokenize{Advanced/Animation::doc}}
\begin{DUlineblock}{0em}
\item[] Vrtist uses a custom animation engine.
\item[] 
\item[] The creation and edition of animations are done by selecting an object or a group of objects and using the dopesheet panel.
\item[] 
\end{DUlineblock}

\noindent{\hspace*{\fill}\sphinxincludegraphics[width=300\sphinxpxdimen]{{Dopesheet}.png}\hspace*{\fill}}

\begin{DUlineblock}{0em}
\item[] 
\item[] Properties that can be animated are: the position, rotation and scale of objects. The intensity and color of lights. The focal, focus and aperture of cameras.
\item[] To open the dopesheet panel go to the camera tool on the palette, and check the animation editor
\item[] 
\item[] The dopesheet offers a variety of controls.
\item[] 
\item[] Interpolation types for animation curves: \sphinxincludegraphics[width=23\sphinxpxdimen]{{ConstantInterpolation}.png} constant, \sphinxincludegraphics[width=23\sphinxpxdimen]{{LinearInterpolation}.png} linear, \sphinxincludegraphics[width=23\sphinxpxdimen]{{BezierInterpolation}.png} Bezier
\item[] Timeline controls:
\item[] \sphinxincludegraphics[width=23\sphinxpxdimen]{{player_gotostart}.png} Go to start.
\item[] \sphinxincludegraphics[width=23\sphinxpxdimen]{{player_prev_keyframe}.png} Go to the previous keyframe.
\item[] \sphinxincludegraphics[width=23\sphinxpxdimen]{{player_prev}.png} Go to the previous frame.
\item[] \sphinxincludegraphics[width=23\sphinxpxdimen]{{player_play}.png} Play animation.
\item[] \sphinxincludegraphics[width=23\sphinxpxdimen]{{player_next}.png} Go to the next frame.
\item[] \sphinxincludegraphics[width=23\sphinxpxdimen]{{player_next_keyframe}.png} Go to the next keyframe.
\item[] \sphinxincludegraphics[width=23\sphinxpxdimen]{{player_gotoend}.png} Go to end.
\item[] \sphinxincludegraphics[width=23\sphinxpxdimen]{{player_record}.png} Record. Clicking on this icon will start a timer. At the end of this timer, move the selection as wanted. This will automatically create keyframes for each frame.
\item[] Click the \sphinxincludegraphics[width=23\sphinxpxdimen]{{player_stop}.png} icon that replaced the record icon to stop recording.
\end{DUlineblock}

\begin{DUlineblock}{0em}
\item[] The \sphinxincludegraphics[width=23\sphinxpxdimen]{{trash}.png} icon will delete the animation for the selection.
\item[] \sphinxincludegraphics[width=23\sphinxpxdimen]{{add_keyframe}.png} Add a keyframe at the current time for the object. Set the object as wanted, then click the icon to add a keyframe. Move the current frame on the timeline, change the object parameters, then add a new keyframe. Repeat these steps as needed to create an animation.
\item[] \sphinxincludegraphics[width=23\sphinxpxdimen]{{auto_keyframe}.png} Auto\sphinxhyphen{}key. Click this icon to lock the auto\sphinxhyphen{}key setting. A keyframe will be added every time the object is moved or a parameter is changed. Move to a new frame, and move the object/parameter to create a new keyframe.
\item[] \sphinxincludegraphics[width=23\sphinxpxdimen]{{remove_keyframe}.png} Remove a keyframe at the current frame.
\item[] To change the current frame, slide the blue marker on the timeline. By default the timeline shows keyframes 0 to 250. These values can be changed by clicking on them.
\item[] To change the displayed keyframes drag the start or end circles on the slider over the timeline. The gap can then be dragged along the slider to change the section displayed.
\item[] 
\end{DUlineblock}

\noindent{\hspace*{\fill}\sphinxincludegraphics[width=300\sphinxpxdimen]{{TimelineSlider}.png}\hspace*{\fill}}

\begin{DUlineblock}{0em}
\item[] 
\item[] The current keyframe can also be changed by moving the primary controller joystick left or right.
\item[] 
\item[] The label on the right of the keyframe shows the current time, and keyframe number.
\item[] Underneath the selected object or number of objects is shown.
\item[] Under the timeline, keyframes added for this selection are displayed by a yellow diamond.
\end{DUlineblock}


\section{Camera Montage \& Editing}
\label{\detokenize{Advanced/Montage:camera-montage-editing}}\label{\detokenize{Advanced/Montage::doc}}
\noindent{\hspace*{\fill}\sphinxincludegraphics[width=300\sphinxpxdimen]{{ShotManager}.png}\hspace*{\fill}}

\begin{DUlineblock}{0em}
\item[] 
\item[] The camera tool can also be used to create a sequence of shots.
\item[] 
\item[] To create a sequence of shots you will need multiple cameras in the scene.
\item[] Start by creating one camera for each wanted shot. These cameras can be animated if needed.
\item[] Then open the camera panel, and tick the shot Manager checkbox to open the shot manager panel. Opening the Animation editor will help the process.
\item[] 
\item[] Select a camera in the scene, then add it to the shot manager by clicking the \sphinxincludegraphics[width=23\sphinxpxdimen]{{add}.png} icon.
\item[] Set the start frame of this shot by moving to the desired frame on the timeline, then clicking the right arrow on the shot line. Set the end frame of the shot in the same way with the left arrow.
\item[] Repeat this process for each camera used in the montage.
\item[] Select a camera and use the up and down arrow at the top of the panel to reorder it.
\item[] To duplicate a shot, select the shot line, and click the \sphinxincludegraphics[width=23\sphinxpxdimen]{{duplicate}.png} icon.
\item[] To deactivate a shot, uncheck the box at the beginning.
\item[] To remove a shot, select the line, and click the \sphinxincludegraphics[width=23\sphinxpxdimen]{{remove}.png} icon.
\item[] 
\item[] To preview the result, tick the “montage” checkbox, open the Camera preview window in the camera panel and play the timeline.
\item[] The shot manager will go to the first frame of the first shot, play until the last frame of the shot, then jump to the first frame of the next shot to play. This will be repeated for each shot.
\item[] Click the \sphinxincludegraphics[width=23\sphinxpxdimen]{{record_video}.png} icon in the shot manager panel to create and export a video of the result.
\end{DUlineblock}


\chapter{Script References}
\label{\detokenize{index:script-references}}

\section{Script References}
\label{\detokenize{ScriptReferences/Script:script-references}}\label{\detokenize{ScriptReferences/Script::doc}}

\subsection{Overview}
\label{\detokenize{ScriptReferences/Script:overview}}
\begin{DUlineblock}{0em}
\item[] VRtist is centered around three main static classes:
\item[] Global State for the core. It has references for settings, player controllers and animation engine.
\item[] Tools Manager for the tools themselves, mostly getting and changing active tools.
\item[] Scene Manager which contains many static methods to interact with the content of the scene.
\item[] 
\item[] For project wide interaction that could impact various parts of the project, (ie: added objects, selection changes…) UnityEvents are used. Most of these events are in the Global State.
\item[] 
\end{DUlineblock}

\noindent{\hspace*{\fill}\sphinxincludegraphics[width=700\sphinxpxdimen]{{Diagrame}.png}\hspace*{\fill}}

\begin{DUlineblock}{0em}
\item[] 
\end{DUlineblock}


\subsection{Tools}
\label{\detokenize{ScriptReferences/Script:tools}}
\begin{DUlineblock}{0em}
\item[] Tools are managed by two static classes:
\item[] Tool Manager for the logic part, referencing all available tools and switching between them.
\item[] Tool UI Manager for the User interface part.
\item[] 
\item[] All tools inherit from the ToolBase class.
\item[] The tool base class registers the tool in the Tool Manager.
\item[] The base class also implements methods for interface interaction (slider, buttons…)
\item[] Tool base also implements DoUpdate and UpdateUI for tools that need to be manually updated.
\item[] Tools that use selection must inherit from SelectorBase instead of ToolBase.
\item[] 
\item[] 
\item[] For the interface:
\item[] Tool Icons use the UIButton script. They are placed in CameraRig/Pivot/PaletteController/Palette/MainPanel.
\item[] They use the OnReleaseEvents to call ToolsUIManager.ChangeTool and ToolsUIManager.ChangeTab with the name of the tool and ToolsUIManager.ShowColorPanel with whether it should show the color picker panel.
\item[] 
\end{DUlineblock}

\noindent{\hspace*{\fill}\sphinxincludegraphics[width=300\sphinxpxdimen]{{UIButtonEvents}.png}\hspace*{\fill}}

\begin{DUlineblock}{0em}
\item[] 
\item[] The tool panel is placed in CameraRig/Pivot/PaletteController/Palette/MainPanel/ToolsPanelGroup and contains the UIPanel Script.
\item[] 
\item[] For the interaction:
\item[] The tool script itself is in CameraRig/Pivot/ToolsController/Tools and the object name is the same that is called in ToolsUIManager.ChangeTool
\item[] 
\end{DUlineblock}


\subsection{Animation engine}
\label{\detokenize{ScriptReferences/Script:animation-engine}}
\begin{DUlineblock}{0em}
\item[] The animation Engine class stores and manages all animations in the scene.
\item[] 
\item[] An animation is represented by an animation set.
\item[] An animation set contains the transform of the animated object and a dictionary of animatable properties with their curves.
\item[] A curve is represented by a list of animation keys (frame, value, interpolation type).
\item[] The curve contains a cache of its values for each frame. When the curve is modified (ie: a keyframe is added or removed) the curve will recalculate it’s cache.
\item[] 
\item[] When the current frame is changed, the animation engine will evaluate every animated object. To do so, for each animation set, it will read the value cached in the curves for the frame, and apply it to the property defined by this curve.
\item[] 
\item[] The animation engine subscribes to two events from Global state. When an object is added to the scene, it will check if that object had a previous animation, and if so compute it’s curves cache. When an object is removed from the scene, it will store the animation set, but free it’s curves cache.
\item[] 
\item[] Animation curves are drawn by the Animation3DCurveManager by reading the cached values of position curves.
\item[] 
\end{DUlineblock}


\subsection{Commands}
\label{\detokenize{ScriptReferences/Script:commands}}
\begin{DUlineblock}{0em}
\item[] Every action in VRtist herits from the ICommand class.
\item[] This allows actions to be undone or redone.
\item[] 
\item[] An action has to implement three methods:
\item[] Redo: the operations done to apply the action.
\end{DUlineblock}

\begin{sphinxVerbatim}[commandchars=\\\{\},numbers=left,firstnumber=84,stepnumber=1]
\PYG{k}{public} \PYG{k}{override} \PYG{k}{void} \PYG{n+nf}{Undo}\PYG{p}{(}\PYG{p}{)}
\PYG{p}{\PYGZob{}}
    \PYG{k+kt}{int} \PYG{n}{count} \PYG{p}{=} \PYG{n}{objects}\PYG{p}{.}\PYG{n}{Count}\PYG{p}{;}
    \PYG{k}{for} \PYG{p}{(}\PYG{k+kt}{int} \PYG{n}{i} \PYG{p}{=} \PYG{l+m}{0}\PYG{p}{;} \PYG{n}{i} \PYG{p}{\PYGZlt{}} \PYG{n}{count}\PYG{p}{;} \PYG{n}{i}\PYG{p}{+}\PYG{p}{+}\PYG{p}{)}
    \PYG{p}{\PYGZob{}}
        \PYG{n}{GameObject} \PYG{n}{ob} \PYG{p}{=} \PYG{n}{objects}\PYG{p}{[}\PYG{n}{i}\PYG{p}{]}\PYG{p}{;}
        \PYG{n}{SceneManager}\PYG{p}{.}\PYG{n}{SetObjectTransform}\PYG{p}{(}\PYG{n}{ob}\PYG{p}{,} \PYG{n}{beginPositions}\PYG{p}{[}\PYG{n}{i}\PYG{p}{]}\PYG{p}{,} \PYG{n}{beginRotations}\PYG{p}{[}\PYG{n}{i}\PYG{p}{]}\PYG{p}{,} \PYG{n}{beginScales}\PYG{p}{[}\PYG{n}{i}\PYG{p}{]}\PYG{p}{)}\PYG{p}{;}
    \PYG{p}{\PYGZcb{}}
\PYG{p}{\PYGZcb{}}
\end{sphinxVerbatim}

\begin{DUlineblock}{0em}
\item[] 
\item[] Undo: the operations done to undo the action.
\end{DUlineblock}

\begin{sphinxVerbatim}[commandchars=\\\{\},numbers=left,firstnumber=93,stepnumber=1]
\PYG{k}{public} \PYG{k}{override} \PYG{k}{void} \PYG{n+nf}{Redo}\PYG{p}{(}\PYG{p}{)}
\PYG{p}{\PYGZob{}}
    \PYG{k+kt}{int} \PYG{n}{count} \PYG{p}{=} \PYG{n}{objects}\PYG{p}{.}\PYG{n}{Count}\PYG{p}{;}
    \PYG{k}{for} \PYG{p}{(}\PYG{k+kt}{int} \PYG{n}{i} \PYG{p}{=} \PYG{l+m}{0}\PYG{p}{;} \PYG{n}{i} \PYG{p}{\PYGZlt{}} \PYG{n}{count}\PYG{p}{;} \PYG{n}{i}\PYG{p}{+}\PYG{p}{+}\PYG{p}{)}
    \PYG{p}{\PYGZob{}}
        \PYG{n}{GameObject} \PYG{n}{ob} \PYG{p}{=} \PYG{n}{objects}\PYG{p}{[}\PYG{n}{i}\PYG{p}{]}\PYG{p}{;}
        \PYG{n}{SceneManager}\PYG{p}{.}\PYG{n}{SetObjectTransform}\PYG{p}{(}\PYG{n}{ob}\PYG{p}{,} \PYG{n}{endPositions}\PYG{p}{[}\PYG{n}{i}\PYG{p}{]}\PYG{p}{,} \PYG{n}{endRotations}\PYG{p}{[}\PYG{n}{i}\PYG{p}{]}\PYG{p}{,} \PYG{n}{endScales}\PYG{p}{[}\PYG{n}{i}\PYG{p}{]}\PYG{p}{)}\PYG{p}{;}
    \PYG{p}{\PYGZcb{}}
\PYG{p}{\PYGZcb{}}
\end{sphinxVerbatim}

\begin{DUlineblock}{0em}
\item[] 
\item[] Submit: Calls redo a first time, and register the action in the CommandManager.
\end{DUlineblock}

\begin{sphinxVerbatim}[commandchars=\\\{\},numbers=left,firstnumber=102,stepnumber=1]
\PYG{k}{public} \PYG{k}{override} \PYG{k}{void} \PYG{n+nf}{Submit}\PYG{p}{(}\PYG{p}{)}
\PYG{p}{\PYGZob{}}
    \PYG{k}{if} \PYG{p}{(}\PYG{k}{null} \PYG{p}{!}\PYG{p}{=} \PYG{n}{objects} \PYG{p}{\PYGZam{}}\PYG{p}{\PYGZam{}} \PYG{n}{objects}\PYG{p}{.}\PYG{n}{Count} \PYG{p}{\PYGZgt{}} \PYG{l+m}{0}\PYG{p}{)}
    \PYG{p}{\PYGZob{}}
        \PYG{n}{Redo}\PYG{p}{(}\PYG{p}{)}\PYG{p}{;}
        \PYG{n}{CommandManager}\PYG{p}{.}\PYG{n}{AddCommand}\PYG{p}{(}\PYG{k}{this}\PYG{p}{)}\PYG{p}{;}
    \PYG{p}{\PYGZcb{}}
\PYG{p}{\PYGZcb{}}
\end{sphinxVerbatim}

\begin{DUlineblock}{0em}
\item[] 
\item[] Command operations go through the scene manager.
\end{DUlineblock}

\begin{sphinxVerbatim}[commandchars=\\\{\},numbers=left,firstnumber=65,stepnumber=1]
\PYG{k}{static} \PYG{k}{readonly} \PYG{n}{List}\PYG{p}{\PYGZlt{}}\PYG{n}{ICommand}\PYG{p}{\PYGZgt{}} \PYG{n}{undoStack} \PYG{p}{=} \PYG{k}{new} \PYG{n}{List}\PYG{p}{\PYGZlt{}}\PYG{n}{ICommand}\PYG{p}{\PYGZgt{}}\PYG{p}{(}\PYG{p}{)}\PYG{p}{;}
\PYG{k}{static} \PYG{k}{readonly} \PYG{n}{List}\PYG{p}{\PYGZlt{}}\PYG{n}{ICommand}\PYG{p}{\PYGZgt{}} \PYG{n}{redoStack} \PYG{p}{=} \PYG{k}{new} \PYG{n}{List}\PYG{p}{\PYGZlt{}}\PYG{n}{ICommand}\PYG{p}{\PYGZgt{}}\PYG{p}{(}\PYG{p}{)}\PYG{p}{;}
\end{sphinxVerbatim}

\begin{DUlineblock}{0em}
\item[] When the action is done the first time, a command is created to store the state before and after the action. Then submit is called on that action.
\item[] When the action is registered in the Command manager, it’s placed at the top of the command stack.
\item[] When the user undoes an action, the Command manager calls the undo method on that action and moves it to the top of the redo stack.
\item[] When the user redo an action, the Command manager calls the redo method on that action and moves it to the top of the undo stack.
\end{DUlineblock}


\subsection{Imports}
\label{\detokenize{ScriptReferences/Script:imports}}
\begin{DUlineblock}{0em}
\item[] Importation of assets into VRtist is done using the Open Asset Import Library (Assimp).
\item[] Asset importation is done over multiple frames to avoid freezes.
\item[] 
\item[] Assets importation takes the following steps:
\item[] First the file is imported with assimp to create an Assimp scene.
\item[] Then materials are read from the assimp scene and recreated with Unity materials.
\item[] Then meshes are read and recreated in Unity.
\item[] A root GameObject is created.
\item[] Finally the hierarchy is recreated and meshes and materials are applied to each object.
\end{DUlineblock}


\subsection{Movable objects}
\label{\detokenize{ScriptReferences/Script:movable-objects}}
\begin{DUlineblock}{0em}
\item[] For an object to be movable it must have a collider set as trigger and be tagged as “PhysicObject”. Object collision detection with the controller is done in SelectorTrigger.
\end{DUlineblock}


\subsection{Inputs}
\label{\detokenize{ScriptReferences/Script:inputs}}
\begin{DUlineblock}{0em}
\item[] User inputs are gathered with the InputDevice class and the TryGetFeature. This allows inputs to be tested anywhere in the project.
\end{DUlineblock}

\begin{sphinxVerbatim}[commandchars=\\\{\},numbers=left,firstnumber=152,stepnumber=1]
\PYG{c+c1}{// Get right controller buttons states}
\PYG{k+kt}{bool} \PYG{n}{primaryButtonState} \PYG{p}{=} \PYG{n}{VRInput}\PYG{p}{.}\PYG{n}{GetValue}\PYG{p}{(}\PYG{n}{VRInput}\PYG{p}{.}\PYG{n}{primaryController}\PYG{p}{,} \PYG{n}{CommonUsages}\PYG{p}{.}\PYG{n}{primaryButton}\PYG{p}{)}\PYG{p}{;}
\PYG{k+kt}{bool} \PYG{n}{triggerState} \PYG{p}{=} \PYG{n}{VRInput}\PYG{p}{.}\PYG{n}{GetValue}\PYG{p}{(}\PYG{n}{VRInput}\PYG{p}{.}\PYG{n}{primaryController}\PYG{p}{,} \PYG{n}{CommonUsages}\PYG{p}{.}\PYG{n}{triggerButton}\PYG{p}{)}\PYG{p}{;}
\end{sphinxVerbatim}



\renewcommand{\indexname}{Index}
\printindex
\end{document}